\chapter{Requerimientos}

\section{Requerimientos de interfaces externas}
\begin{flushright}
	
	\begin{enumerate} [label=RF-\arabic*. , align=left]
		\setcounter{enumi}{200}	 % Los requerimientos empiezan con el valor 200
		\item El dispositivo debera tener conexion a internet, ya sea mediante una red Wifi o mediante el uso de datos moviles.
		
		\item El dispositivo tendra que tener activado el Gps para mayor facilidad de la ubicacion de mascotas.
		\item 
		
		
	\end{enumerate}
	
\end{flushright}


\section{Requerimientos Funcionales}

A continuación se detalla los requerimientos funcionales de la aplicación organizado por módulos:

\subsection{Creacion de Cuenta}
\begin{table}[h!]	
	\begin{tabular}{ |p{2cm}|p{12cm}| }	\hline
		
		\rowcolor{gray!50}  \textbf{No.}  &  \textbf{Descripción} \\ \hline
		
		RF-201&Los usuarios deberan crear su cuenta,se le pedira los siguientes requesitos: nombre, apellidos, correo electronico, direccion, edad, numero de celular.  \\	\hline
		
		RF-202& Por otro lado tambien podra registrarse con la autenticacion de su correo, en este caso solo completara algunos datos como direccion,numero de celular. \\ \hline
		
		RF-203&El usuario poodra modificar sus datos como direccion, correo electronico, edad, numero de celular.\\ \hline
		
		RF-204& Solo se podran registrar personas mayores a 18 años previa validacion.  \\ \hline
		
		
	\end{tabular}
\end{table}	




\subsection{Inicio de Sesión}
	\begin{table}[h!]	
		\begin{tabular}{ |p{2cm}|p{12cm}| }	\hline
			
			\rowcolor{gray!50}  \textbf{No.}  &  \textbf{Descripción} \\ \hline
			
			RF-205&Los usuarios deberan tener una credencial compuesta por un USUARIO y su CLAVE.  \\	\hline
			
			RF-206&El clave debe ser un número de 6 digitos que debera ser cambiada cada 6 meses.  \\ \hline
			
			RF-207&El usuario tendra la posibilidad de visualizar el perfil de las mascotas registradas.Tambien podra adicionar nuevas mascotas,incluso la eliminacion de los perfiles creados(mascotas).  \\ \hline
		
			
			RF-207&El usuario podra cerrar sesion.  \\ \hline
				
		
		\end{tabular}
	\end{table}	


\subsection{Registro de mascotas abandonadas}

\begin{flushright}

	\begin{enumerate} [label=RF-\arabic*. , align=left]
		\setcounter{enumi}{200}	 % Los requerimientos empiezan con el valor 200
		\item Los usuarios podran editar el perfil de la mascota abandonada,colocando una imagen, una descripcion breve(caracteristicas de la mascota), tambien la ubicacion donde lo encontraron.
		 
		
	\end{enumerate}

\end{flushright}
\subsection{Registro de mascotas perdidas}

\begin{flushright}
	
	\begin{enumerate} [label=RF-\arabic*. , align=left]
		\setcounter{enumi}{200}	 % Los requerimientos empiezan con el valor 200
		\item Los usuarios podran editar el perfil de la mascota perdida,colocando una imagen, tamaño, raza,lugar donde se perdio, contacto.
		
	\end{enumerate}
	
\end{flushright}


\section{Requerimientos No Funcionales}
A continuación se detalla los requerimientos no funcionales de la aplicación:
\begin{flushright}
	
	\begin{enumerate} [label=RF-\arabic*. , align=left]
		\setcounter{enumi}{200}	 % Los requerimientos empiezan con el valor 200
		\item La informacion manipulados por la aplicacion no deben contener informacion sensible del usuario o mascotas.
	  
		\item Algún tipo de transacción monetaria dentro de la aplicación que involucren al usuario.
		 
		\item Garantía de la información obtenida de las marcas y empresas relacionadas a la aplicación.
		 
		
	\end{enumerate}
	
\end{flushright}



